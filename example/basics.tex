\chapter{Basics}
\begin{multicols}{2}
\section{The Basics}

\subsection{What is Warp Lords?}
Warp Lords is a tabletop role-playing game (TTRPG). TTRPGs are games that let people play pretend. However, they still have rules for what characters can do. This prevents that one person in your group from going, "the dragon doesn't kill me because I have a force field!"

Games are run by a Game Master (GM) who functions as the one directing the story. The players are the main characters of the story. It is the GM's job to set the scene, control the enemies that players fight, and roleplay the Non-Player Characters (NPCs) that the players interact with.

You might be familiar with more popular TTRPGs. You may have played some before. This game may have some things in common with other TTRPGs, but there's nothing out there quite like it.

\subsection{What’s The Game Like?}
Warp Lords takes place in a vast multiverse where anything is possible. The rules were built from the ground up to be flexible so it can cover pretty much any genre you can think of. In the course of a single session, you could fight a dragon, steal a space ship, visit a magic school, and have a musical number.

The mechanics are simple enough for newcomers but have enough depth to keep veteran players engaged. There's a strong emphasis on customization. The XP-based progression system lets you build your character the way you want without those extra abilities you'll never use. If you want to specialize in one thing, you can. If you want to be a jack-of-all-trades, you can.

Warp Lords is a d12 (12-sided dice) based system that uses tarot cards, and has flexible rules in and out of combat. It may seem weird at first, but it doesn't take long at all to get used to. Once you get the hang of it, the game can flow pretty quickly.

\subsection{What’s With the Name?}
As mentioned previously, the game takes place in a multiverse. So you can in fact warp between universes. The players also have the potential to ascend to godhood and gather followers, essentially becoming lords, who happen to warp between realities. Lords who warp. Warp Lords. We didn't come up with the name first and justify it after. That's definitely not what happened.

\subsection{What Do You Need?}
\begin{itemize}
\item Five to ten 12-sided dice (d12). More for high-level games.
\item A deck of tarot cards
\item A pencil
\item An eraser
\item Paper
\end{itemize}

\subsection{Core Mechanics}
Characters have ten stats each, ranging from 1-12. These stats determine how powerful a character’s skills are. Skills are special abilities that can be purchased based on a character’s race, as well as class. They also factor into when and how many dice you roll in a given situation. For more details, see the sections for stats, race, and class.

\subsection{Blackjack Grenade Engine}
Warp Lords uses a unique dice system called the Blackjack Grenade Engine (BGE). The BGE system uses pools of exploding dice for opposed rolls and a roll-under system for unopposed rolls. All rolls are divided into opposed and unopposed rolls.

\subsection{Opposed Rolls (Dice Pool)}
An opposed roll represents a contest between two or more characters. An example of this would be a combat or social encounter. The numbers of dice you roll for this encounter are dictated by your stats, or your weapon/shield rank, and any relevant bonuses get added as well. After all dice have been rolled, the person who rolled the highest number wins.

\subsubsection{Exploding Dice:}
On a 12, you reroll the die and add that result to the initial roll. If you roll another 12, repeat the process. Do so until you roll a non-explosion (1-11). Each die, including the initial 12, count towards the total number of your roll.

\subsection{Unopposed Rolls (Blackjack)}
Unopposed rolls represent something a character could fail at, but that isn’t a contest with another character.  This could include something such as trying to lift an object, or something as simple as noticing an event that occurs in the world. For whichever stat is being represented, roll 1d12 and try to get as high as possible a number without exceeding your total in the relevant stat. Rolling under represents a success, but rolling over the number represents a failure.

\subsubsection{Blackjack: }
A blackjack occurs when you roll the exact number of the given stat, and represents a critical success. 

\subsection{Character Progression}
All stat increases, skills, racial abilities, and additional classes are purchased using experience points (XP). Characters get a starting pool of XP to build their characters at character creation.

\noindent Starting XP is as follows:
\begin{itemize}
\item Low Power: 50XP
\item Normal Power: 100XP
\item High Power: 150XP
\end{itemize}

\note{100XP is the recommended starting point!}

%The game master (GM) hands out XP at the end of each session. They can hand out any amount of XP they deem appropriate, but here are some guidelines.
%\begin{itemize}
%\item Slow Progression: 5XP
%\item Normal Progression: 10XP
%\item Fast Progression 15XP
%\end{itemize}

%\note{The most powerful skills cost 15XP and the highest a standard stat increase can cost is 12XP. \\Handing out 10XP per session for a typical campaign lets players increase low stats or purchase cheap skills each session, but they need to save up for more expensive skills or stat increases.}
\begin{paperbox}{Awarding XP}
	The game master (GM) hands out XP at the end of each session. They can hand out any amount of XP they deem appropriate, but here are some guidelines.
	\begin{itemize}
	\item Slow Progression: 5XP
	\item Normal Progression: 10XP
	\item Fast Progression 15XP
	\end{itemize}
	The most powerful skills cost 15XP and the highest a standard stat increase can cost is 12XP. \\
	Handing out 10XP per session for a typical campaign lets players increase low stats or purchase cheap skills each session, but they need to save up for more expensive skills or stat increases.
\end{paperbox}
	
\note[REMEMBER:]{The listed amounts for starting XP and per-session XP are just suggestions. It's ultimately up to the GM to decide what's best for their group.}
	
\subsection{XP Alternatives}
Instead of giving players more XP, GMs can give players alternate rewards based on roleplay and campaign events. These rewards can be anything they deem fitting, including, but not limited to, the following:
 
\begin{itemize}
\item Awarding certain skills and paths their class normally wouldn’t have access to.
\item Directly giving players skills or stat increases.
\item Giving players special pets or NPC followers.
\item Creating unique skills for the players. 
\item Anything they deem fitting to reward their players.
\end{itemize}

\section{The Stats}
There are ten different categories of statistic with which to spend your points. Each focuses on different areas of your characters development.
% statdescription command: parameters: 1 stat name, 2 description, 3 example
\newcommand{\statdescription}[3]{
	\begin{samepage}
	\subsection{#1:}
	#2\\
	\textbf{Example: }#3
	\end{samepage}
}
\statdescription{\STRstat{} (\strstat)}{ A character’s physical strength. Used for melee damage, and tests of strength.}{Opposed rolls to grapple foe. Unopposed check to lift a heavy object.}
\statdescription{\VITstat{} (\vitstat)}{A character’s health and stamina. Used for wounds, immune system, and endurance.}{Opposed rolls for a drinking contest. Unopposed check to resist hangover.}
\statdescription{\AGIstat{} (\agistat)}{ A character’s speed and dexterity. Used for speed, dodging, aiming, and acrobatics.}{Opposed rolls for a foot race. Unopposed check to balance on a tightrope.}
\statdescription{\INTstat{} (\intstat)}{A character’s memory, knowledge, and speed of thought. Used for knowledge checks, scholar abilities, and known languages (+1 per rank).}{Opposed rolls for strategy games. Unopposed check to remember something.}
\statdescription{\CHAstat{} (\chastat)}{A character’s social skills, prowess, and force of personality. Used for charm, manipulation, and social contests.}{ Opposed rolls to persuade someone. Unopposed check to gather a crowd.}
\statdescription{\STYstat{} (\stystat)}{A character’s creativity and flair. Use for crafting and bard abilities.}{Opposed rolls for a pose-off. Unopposed check to craft objects.}
\statdescription{\WILstat{} (\wilstat)}{A character’s concentration, faith, and mental fortitude. Used for resisting mental effects, focus, prayer, and priest abilities.}{Opposed rolls to resist an illusion. Unopposed check for a god to hear your prayer.}
\statdescription{\SENstat{} (\senstat)}{A character’s perception and common sense. Used for noticing things.}{Opposed rolls to notice someone hiding. Unopposed check to hear a kerfuffle.}
\statdescription{\MAGstat{} (\magstat)}{A character’s aura and mana capacity. Used for inherent supernatural abilities and channeling spells.}{Opposed rolls to cast a spell. Unopposed check to identify magic items.}
\statdescription{\LUCstat{} (\lucstat)}{A character’s defense against random tragedy. Used for games of chance and gambler abilities. And the occasional lottery.}{Opposed rolls for a game of chance. Unopposed check to avoid an accident.}

\section{How stats work}
All basic stats fall under a range of 1-12. In the beginning all stats default to a 2. Race and class will dictate extra bonuses before you begin spending XP. Stats can also be temporarily raised and lowered by various means (See classes and combat sections). Stats are broken into 5 levels, or tiers, called “Ranks”. These ranks determine the number of d12’s in your dice pool when it comes time to make an opposed roll. This dice pool is called “Rank Dice” (RD). See table 2-1 for a breakdown of ranks.

\subsection{Stat:}
A character’s true stat, unmodified by buffs or damage.

\subsection{Rank:}
An indicator of the strength of a given stat. Determines how many dice you use on an opposed roll. These are referred to as “Rank Dice” and are abbreviated as (RD) in skill descriptions.
TODO

\subsection{Temporary Stat:}
A stat that is temporarily raised or lowered for any reason.

\subsection{Stat Bonus:}
Stat Bonuses are bonuses added to Rank Dice rolls. Simply add the relevant stat to the total roll. These bonuses typically come from skills. This is abbreviated as (SB) in skill descriptions.
\begin{wltable}[XXX]<Skill Ranks><tab:skillrank><
Stat & Rank & Rank Dice>[12]
1-2  &\  E (1) & 1d12\\
3-5  &\  D (2) & 2d12\\
6-8  &\  C (3) & 3d12\\
9-11 &\  B (4) & 4d12\\
12   &\  A (5) & 5d12\\
\end{wltable}
\vspace{4\baselineskip}
%\vspace{12pt}
\subsection{EX Stats}
Stats can be pushed past the usual boundary of 12 and can in fact go as high as 24. Anything beyond a 12/A is referred to as an “EX” stat. Rather than adding more dice to your opposed rolls, EX stats instead increase the explosion range of a given stat’s rank dice. However, stats that exceed 12 are recommended for high level campaigns or special circumstances, such as a character ascending to godhood, or beginning the process of doing so. See Table~\ref{tab:exskillrank} for a breakdown of EX ranks.
\begin{wltable}[XXc]<EX Stats Explosion Range><tab:exskillrank><
Stat & Rank & Explosion Range>[6]
13-16 & EX 1 (6) & 11-12\\
17-20 & EX 2 (7) & 10-12\\
21-24 & EX 3 (8) & 9-12\\
\end{wltable}

For the actual total cost of spending XP to attain certain ranks, including EX ranks, see Tables~\ref{tab:skillcost}, and \ref{tab:exskillcost}.
\vspace{4\baselineskip}
\end{multicols}

\subsection{XP Cost Breakdown}
These tables represent the total XP costs of bringing one stat from one level to a new, higher level. Table~\ref{tab:skillcost} covers the basic 12 levels, while Table~\ref{tab:exskillcost} covers the cost of EX ranks.

\begin{wltable}[Y[2]YYYYYYYYYYY]<Stats Cost><tab:skillcost><
Starting Stat
   & 2 & 3 & 4 & 5  & 6  & 7  & 8  & 9  & 10 & 11 & 12>
1  & 2 & 5 & 9 & 14 & 20 & 27 & 35 & 44 & 54 & 65 & 77 \\
2  & X & 3 & 7 & 12 & 18 & 25 & 33 & 42 & 52 & 63 & 75 \\
3  & X & X & 4 & 9  & 15 & 22 & 30 & 39 & 49 & 60 & 72 \\
4  & X & X & X & 5  & 11 & 18 & 26 & 35 & 45 & 56 & 68 \\
5  & X & X & X & X  & 6  & 13 & 21 & 30 & 40 & 51 & 63 \\
6  & X & X & X & X  & X  & 7  & 15 & 24 & 34 & 45 & 57 \\
7  & X & X & X & X  & X  & X  & 8  & 17 & 27 & 38 & 50 \\
8  & X & X & X & X  & X  & X  & X  & 9  & 19 & 30 & 42 \\
9  & X & X & X & X  & X  & X  & X  & X  & 10 & 21 & 33 \\
10 & X & X & X & X  & X  & X  & X  & X  & X  & 11 & 23 \\
11 & X & X & X & X  & X  & X  & X  & X  & X  & X  & 12 
\end{wltable}
\begin{wltable}[Y[2]YYYYYYYYYYYY]<EX Stats Cost><tab:exskillcost><
Starting Stat
   & 13 & 14 & 15 & 16 & 17 & 18 & 19  & 20  & 21  & 22  & 23  & 24>
12 & 13 & 27 & 42 & 58 & 75 & 93 & 112 & 132 & 153 & 175 & 198 & 222  \\
13 & X  & 14 & 29 & 45 & 62 & 80 & 99  & 119 & 140 & 162 & 185 & 209  \\
14 & X  & X  & 15 & 31 & 48 & 66 & 85  & 105 & 126 & 148 & 171 & 195  \\
15 & X  & X  & X  & 16 & 33 & 51 & 70  & 90  & 111 & 133 & 156 & 180  \\
16 & X  & X  & X  & X  & 17 & 35 & 54  & 74  & 95  & 117 & 140 & 164  \\
17 & X  & X  & X  & X  & X  & 18 & 37  & 57  & 78  & 100 & 123 & 147  \\
18 & X  & X  & X  & X  & X  & X  & 19  & 39  & 60  & 82  & 105 & 129  \\
19 & X  & X  & X  & X  & X  & X  & X   & 20  & 41  & 63  & 86  & 110  \\
20 & X  & X  & X  & X  & X  & X  & X   & X   & 21  & 43  & 66  & 90   \\
21 & X  & X  & X  & X  & X  & X  & X   & X   & X   & 22  & 45  & 69   \\
22 & X  & X  & X  & X  & X  & X  & X   & X   & X   & X   & 23  & 47   \\
23 & X  & X  & X  & X  & X  & X  & X   & X   & X   & X   & X   & 24 
\end{wltable}
\begin{multicols}{2}
\section{Classes}
\begin{quotebox}
Screw party balance.\\\noindent
Go for whatever appeals to you.
\end{quotebox}
At character creation, choose from 1 of 4 classes. A character’s class determines the abilities you will have access to. They grant the following benefits.
\subsection{Stat Bonus}
You gain a +3 to a single stat that is determined by your class.
\subsection{Paths}
Abilities are divided into a number of categories per class referred to as "Paths". Each path acts as a theme for the available abilities such as a path focused on ranged combat or elemental magic. Paths otherwise have no mechanical effect. All paths tied to a certain class are open to you once you select your class.
\subsection{Abilities}
A character’s abilities are their learned skills. Abilities cover a wide range of things from something as basic as cooking to more elaborate fields such as martial arts or spells.

The ability’s strength is determined by its relevant stats.

\subtitlesection{Bonus Ability:} {All classes gain a free bonus ability when selected.}

\subsection{The Classes}
\begin{wltable}[XX]<Class List><tab:classlist><
Class	& Description>
Mage	& Good at magic and casting the spells. \\
Priest	& Good at praying to stuff to get the free powers and stuff. \\
Scholar	& Good at the book learning and the talking to people. \\
Warrior	& Good at the fighting and the weapons and the tactics. \\
\end{wltable}
\subtitlesection{Purchasing Additional Classes}
{You can purchase additional classes for 10XP per additional class. Extra classes grant access to all of the paths abilities within, but you \textbf{\emph{do not}} receive the initial stat bonuses tied to a new class.}
\end{multicols}
